 \documentclass[a4,a4paper,10pt,notitlepage,english]{article}
%%%%%%%%%%%%%%%%%%%%%%%%%%%%%%%%%%%%%%%%%%%%%%%%%%%%%%%%%%%%%%%%%%%%%%
\usepackage{fancyhdr}
\usepackage{amsmath}
\usepackage{amssymb}
\usepackage{amsthm}
\usepackage{enumerate}
\usepackage{fancyvrb,moreverb,boxedminipage}
\usepackage{a4}
%\usepackage{chicago}
\usepackage{graphicx}
\usepackage{geometry}
\usepackage{float}
\usepackage[english]{babel}
% \usepackage[french]{babel}
\usepackage[scaled]{helvet}
\renewcommand{\familydefault}{\sfdefault}
\usepackage[T1]{fontenc}
\usepackage{bm} % makes pdf look better
%
\usepackage[english]{babel}
\usepackage[pdftex,colorlinks=true,
pdfstartview=FitV,
linkcolor=blue,
citecolor=blue,
urlcolor=blue]{hyperref}
%
\tolerance=1
\emergencystretch=\maxdimen
\hyphenpenalty=10000
\hbadness=10000
%
\textwidth=6in
\textheight=8in
\topmargin=0.0in
\oddsidemargin=0in 
\baselineskip=18pt
\headheight=2cm
\pagestyle{fancy}
%
\newcommand{\ds}{\displaystyle}
\newcommand{\bs}[1]{\ensuremath{\boldsymbol{#1}}}
\renewcommand{\leq}{\leqslant}
\newenvironment{refer} 
{
	\begin{list}
		{}
		{
			\setlength{\labelwidth}{.5em}
			\setlength{\leftmargin}{0.4cm}
			\setlength{\itemsep}{0cm}
		} 
	}
	{\end{list}}
%
\theoremstyle{definition}
\newtheorem{exo}{Exercise}
\newtheorem{sol}{Solution}
%
%%%%%%%%%%%%%%%%%%%%%%%%%%%%%%%%%%%%%%%%%%%%%%%%%%%%%%%%%%%%%%%%%%%%%%%%%%%%%%%%%%%%
\lhead{GSEM, University of Geneva\\ \textbf{Forecasting with Applications in Business (`S411031')}\\ Prof. Jeffrey N\"af}
\rhead{Fall Semester 2025 \\ \textbf{Practical 1}\\ Lionel Voirol}
%

\setlength{\parindent}{0em}
%%%%%%%%%%%%%%%%%%%%%%%%%%%%%%%%%%%%%%%%%%%%%%%%%%%%%%%%%%%%%%%%%%%%%%

\begin{document}
	
	\begin{center}
		\fbox{\Large Introduction to Time Series in \texttt{R}}
	\end{center}
	\bigskip
	%%%%%%%%%%%%%%%%%%%%%%%%%%%%%%%%%%%%%%%%%%%%%%%%%%%%%%%%%%%%%%%%%%%%%%%%%%%%%%%%

	
	
	
	
	
	
	
\begin{exo}
    \textbf{Exploration of a built-in \texttt{tsibble} object}\\

    Load the \texttt{tsibble} package (install it if necessary):
    \begin{verbatim}
        > install.packages("tsibble")
        > library(tsibble)
        > library(dplyr)
    \end{verbatim}

    Work on the built-in dataset \texttt{PBS} (Pharmaceutical Benefit Scheme):
    \begin{itemize}
        \item Inspect the first rows of the dataset.
        \item Identify the \texttt{index} (time) and \texttt{key} (grouping variables) of the \texttt{tsibble}.
        \item Find the first and last time points in the dataset.
        \item Count the number of observations per key (e.g., per ATC1 code).
    \end{itemize}
    \smallskip
    Useful resources:
    \begin{itemize}
        \item \href{https://otexts.com/fpp3/tsibbles.html}{\texttt{tsibble} objects in \texttt{fpp3}}
        \item \href{https://tsibble.tidyverts.org/}{\texttt{tsibble} package documentation}
    \end{itemize}
\end{exo}

\bigskip

\begin{exo}
    \textbf{Creating a \texttt{tsibble} from scratch}\\

    Create a small dataset of monthly sales for two products:
    \begin{itemize}
        \item Construct a \texttt{tibble} with columns: \texttt{month}, \texttt{product}, \texttt{sales}.
        \item Convert the \texttt{tibble} into a \texttt{tsibble}, specifying the \texttt{index} and \texttt{key}.
        \item Inspect your \texttt{tsibble}.
        \item Plot sales over time for each product using \texttt{ggplot2}.
    \end{itemize}
    \smallskip
    Hint:
    \begin{verbatim}
        sales_tsibble <- sales_data %>%
          as_tsibble(index = month, key = product)
    \end{verbatim}
\end{exo}

\bigskip
\newpage

\begin{exo}
    \textbf{Creating a \texttt{tsibble} from an external file: Smartphone Sales}\\

    In this exercise, you will work with a dataset containing monthly sales of two smartphone models.
    \begin{itemize}
        \item The file \texttt{smartphone\_sales.csv} contains three columns: \texttt{month}, \texttt{model}, and \texttt{units\_sold}.
        \item Read the CSV file into R using \texttt{readr::read\_csv()}.
        \item Convert the dataset into a \texttt{tsibble}, specifying the \texttt{index} (\texttt{month}) and \texttt{key} (\texttt{model}).
        \item Inspect the tsibble to ensure the time index and keys are correctly set.
        \item Optional: Plot the monthly sales over time for each model using \texttt{ggplot2}.
    \end{itemize}
\end{exo}
	
	
	%%%%%%%%%%%%%%%%%%%%%%%%%%%%%%%%%%%%%%%%%%%%%%%%%%%%%%%%%%%%%%%%%%%%%%%%%%%%%%%%%%
	%%% Exercise 2
	
	%The following exercises are adapted from \textbf{\href{https://otexts.com/fpp3/graphics-exercises.html}{Forecasting: Principles and Practice}} book, Chapter 2. 
	\smallskip
	\begin{exo} 
		Time series visualisation. Part 1.
		\begin{enumerate}
			\item Use the help function to explore what the object \texttt{gafa\_stock}, \texttt{PBS}, \texttt{vic\_elec} and \texttt{pelt} represent.
			
			\begin{enumerate}
				\item Use autoplot() to plot some of the series in these data sets.
				\item What is the time interval of each element?
			\end{enumerate}
			\item Use filter() to find what days correspond to the peak closing price for each of the four stocks in gafa\_stock.
			\item The \texttt{USgas} package contains data on the demand for natural gas in the US.
			
			\begin{enumerate}
				\item Install the \texttt{USgas} package.
				\item Create a tsibble from \texttt{us\_total} with year as the index and state as the key.
				\item Plot the annual natural gas consumption by state for the New England area (comprising the states of Maine, Vermont, New Hampshire, Massachusetts, Connecticut and Rhode Island).
				\item Comment on your findings
			\end{enumerate}
			
		\end{enumerate}
	\end{exo}
	\bigskip
	
	%%%%%%%%%%%%%%%%%%%%%%%%%%%%%%%%%%%%%%%%%%%%%%%%%%%%%%%%%%%%%%%%%%%%%%%%%%%%%%%%%%
	%%% Exercise 3
	
	\begin{exo} 
		Time series visualisation. Part 2.
		\begin{enumerate}
			\item Create time plots of the following four time series: Bricks from \texttt{aus\_production}, Lynx from \texttt{pelt}, Close from \texttt{gafa\_stock}, Demand from \texttt{vic\_elec}.
			\begin{enumerate}
				\item Use ? (or help()) to find out about the data in each series.
				\item Modify the axis labels and titles if needed.
				\item Can you identify any unusual observations?
			\end{enumerate}
			\item What can you conclude? Provide detailed comment on each time series based on your plots.
		\end{enumerate}
		
		
	\end{exo}
	
	%%%%%%%%%%%%%%%%%%%%%%%%%%%%%%%%%%%%%%%%%%%%%%%%%%%%%%%%%%%%%%%%%%%%%%%%%%%%%%%%%%
	%%% Exercise 5
	
	%\begin{exo}
	%Time series decomposition
	
	%Consider the last five years of the Gas data from aus\_production.
	
	%gas <- tail(aus\_production, 5*4) |> select(Gas)
	%\begin{enumerate}
	%\item Plot the time series. Can you identify seasonal fluctuations and/or a trend-cycle?
	%\item Use classical\_decomposition with type=multiplicative to calculate the trend-cycle and seasonal indices.
	%\item Do the results support the graphical interpretation from part a?
	%\item Compute and plot the seasonally adjusted data.
	%\item Change one observation to be an outlier (e.g., add 300 to one observation), and recompute the seasonally adjusted data. What is the effect of the outlier?
	%\item Does it make any difference if the outlier is near the end rather than in the middle of the time series?
	
	%\end{enumerate}
	%\end{exo}
	
	%%%%%%%%%%%%%%%%%%%%%%%%%%%%%%%%%%%%%%%%%%%%%%%%%%%%%%%%%%%%%%%%%%%%%%%%%%%%%%%%%%
	%%% Exercise 6
	%\begin{exo}
	%\begin{enumerate}
	
	%\item Recall your retail time series data (from Exercise 8 in Section 2.10). Decompose the series using X-11. Does it reveal any outliers, or unusual features that you had not noticed previously?
	
	%\end{enumerate}
	%\end{exo}
	%\bigskip
	%%%%%%%%%%%%%%%%%%%%%%%%%%%%%%%%%%%%%%%%%%%%%%%%%%%%%%%%%%%%%%%%%%%%%%%%%%%%%%%%
	%%% Exercise 4
	%\begin{exo}
	%Dependance in Time Series, ACF and PACF plot
	%\begin{itemize}
	%\item Read the chapter 2.3 and 3.1 in the book \textbf{\href{https://smac-group.github.io/ts/}{Applied Time Series Analysis with R}}.
	%\item Read the chapter 2.8 in the book \textbf{\href{https://otexts.com/fpp2}{Forecasting: Principles and Practice}}.
	%\item Read the following article \textbf{\href{https://towardsdatascience.com/significance-of-acf-and-pacf-plots-in-time-series-analysis-2fa11a5d10a8}{Significance of ACF and PACF Plots In Time Series Analysis}} in the online publication \textit{Towards Data Science}
	%\item Simulate AR(p) and MA(p) processes using previously used \texttt{simts} package and plot the corresponding ACF and PACF plot. Be sure to understand the impact of the order p of the process and the changes in the corresponding ACF and PACF plots.
	%\end{itemize}
	%\end{exo}
	%%%%%%%%%%%%%%%%%%%%%%%%%%%%%%%%%%%%%%%%%%%%%%%%%%%%%%%%%%%%%%%%%%%%%%%%%%%%%%%%
\end{document}



	%%%%%%%%%%%%%%%%%%%%%%%%%%%%%%%%%%%%%%%%%%%%%%%%%%%%%%%%%%%%%%%%%%%%%%%%%%%%%%%%%%
	%%% Exercise 4
	\begin{exo}
		Time Series decomposition\\
		
		In the online book \textbf{\href{https://otexts.com/fpp3/components.html}{Forecasting: Principles and Practice}}, within the 'Time series decomposition' section, read the following:
		\begin{itemize}
			\item Chapter 3.2 about 'Time Series components'
			\item Chapter 3.3 about 'Moving averages'
			\item Chapter 3.4 about 'Classical decomposition'
			\item Chapter 3.6 about 'STL decomposition'
		\end{itemize}
	\end{exo}
	\bigskip